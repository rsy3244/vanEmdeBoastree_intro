\documentclass[main]{subfiles}

\begin{document}

\begin{frame}\frametitle{直接アドレス法\uniqnum{2}}
	空間計算量$O(u)$で動的集合を保持する手段として, \structure{直接アドレス法}を考える.\\
\begin{minipage}[t][.4\textheight][c]{\linewidth}
\onslide*<1>{
\begin{block}{直接アドレス法}
	要素の値を配列の添字として利用し, データを保持するテクニック
\end{block}
今回考えているデータ構造では付属データは持たないので, \\各配列の要素には要素を保持しているかをbitで格納する.
}
\onslide*<2->{
	\begin{itemize}
		\item $\Member(V,x)$, $\Insert(V,x)$, $\Delete(V,x)$ の処理は, 配列のランダムアクセスが定数時間であるので, 時間計算量\structure{$O(1)$}. \\
		\item $\Successor(V,x)$の処理は, $x$の次の値からbitが立っている\\要素まで最悪$\Theta(u)$回探索する必要があるので, 時間計算量\alert{$O(u)$}.
			\begin{itemize}
				\item $\Predecessor(V,x)$, $\Min(V)$, $\Max(V)$も同様の処理を行うため, \\時間計算量$O(u)$.
			\end{itemize}
	\end{itemize}
}
\end{minipage}
\begin{minipage}[c][.3\textheight][c]{\linewidth}
\subfile{resources/bt-array.tex}
\end{minipage}
\onslide*<2>{
\begin{tikzpicture}[overlay, remember picture]
	%\node [ellipse callout,draw,callout absolute pointer=(5.5,2.4)](clo) at (3, 5) {$Successor(V,5)$のときの探索};
	\draw[->, thick, red] (4.75,2.33) to [bend right=-45] (5.25,2.33)%
		to [bend right=-45] (5.75,2.33) to [bend right=-45] (6.25,2.33);
\end{tikzpicture}
}
\end{frame}
\end{document}
