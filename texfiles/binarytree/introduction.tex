\documentclass[main]{subfiles}

\begin{document}

\begin{frame}\frametitle{直接アドレス法\uniqnum{2}}
	空間計算量$O(u)$で動的集合を保持する手段として, \structure{直接アドレス法}を考える.\\
\begin{minipage}[c][.6\textheight][c]{\linewidth}
\onslide*<1>{
\begin{block}{直接アドレス法}
	要素の値を配列の添字として利用し, データを保持するテクニック
\end{block}
今回考えているデータ構造では付属データは持たないので, \\各配列の要素には要素を保持しているかをbitで格納する.
}
\onslide*<2>{
	
}
\end{minipage}
\subfile{resources/bt-array.tex}
\end{frame}

\end{document}
