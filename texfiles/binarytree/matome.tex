\documentclass[main]{subfiles}

\begin{document}

\begin{frame}\frametitle{前半のまとめ}
	\begin{itemize}
		\item vEB木を定義した.
		\item 二分木構造で各種操作の時間計算量は下表となった.
		\item 平方分割木によって二分木構造の木の高さを小さくしたが, \\
			一部操作の最悪時間計算量が$O(\sqrt{u})$と悪化\\
			\begin{itemize}
				\item $summary$, $cluster$の線形探索がボトルネック
			\end{itemize}
	\end{itemize}
	\begin{table}
		\begin{tabular}{|c|c|c|}\hline
			& 二分木 & 平方分割木 \\\hline
			{\scriptsize $\Min(V)$ } 			& \structure{ $\Theta(\log u)$} 		& \alert{$O(\sqrt{u })$} \\
			{\scriptsize $\Max(V)$ } 			& \structure{ $\Theta(\log u)$} 		& \alert{$O(\sqrt{u} )$} \\
			{\scriptsize $\Member(V,x)$ } 		& $O(1)$ 					& $O(1)$ \\
			{\scriptsize $\Successor(V,x)$ } 	& \structure{ $O(\log u)$} 	& \alert{$O(\sqrt{u})$} \\
			{\scriptsize $\Predecessor(V,x)$ } 	& \structure{ $O(\log u)$}	& \alert{$O(\sqrt{u})$} \\
			{\scriptsize $\Insert(V,x)$ } 		& \alert{$O(\log u)$}		& \structure{$O(1)$} \\
			{\scriptsize $\Delete(V,x)$ } 		& \structure{ $O(\log u)$}	& \alert{ $O(\sqrt{u})$} \\\hline
		\end{tabular}
	\end{table}
\end{frame}

\end{document}
