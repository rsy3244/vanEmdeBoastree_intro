\documentclass[main]{subfiles}

\begin{document}

\begin{frame}\frametitle{二分木構造\uniqnum{1}}
$Successor(V,x)$で必要な探索回数を小さくするために,\\ 配列の各要素を葉とした\structure{二分木構造}を考える.\\
\onslide*<1>{
\begin{block}{二分木}
	二分木とは, 葉ではない頂点の子が常に2個であるような木
	\begin{description}
		\setlength{\labelwidth}{8ex}
		\item[木] 閉路を持たない, 連結なグラフ\\
		\item[根] 木の頂点のうちの1つに定義する\\
		\item[子] ある頂点について, 隣接する頂点のうち根から遠いもの\\
		\item[葉] 木の頂点のうち, 子を持たないもの\\
	\end{description}
\end{block}
}
\onslide*<2>{
	葉ではない各頂点には, 子の値の論理和を格納する.
	\subfile{resources/binarytree}
}



\end{frame}

\end{document}
