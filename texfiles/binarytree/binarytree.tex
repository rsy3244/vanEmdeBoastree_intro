\documentclass[main]{subfiles}

\begin{document}

\begin{frame}\frametitle{二分木構造\uniqnum{3}}
\onslide*<1>{$\Successor(V,x)$で必要な探索回数を小さくするために,\\ 配列の各要素を葉とした\structure{二分木構造}を考える.\\\vspace{1em}}
\onslide*<1>{
\begin{block}{二分木}
	二分木とは, 葉ではない頂点の子が常に2個であるような木
	\begin{description}[木]
		\item[木] : 閉路を持たない, 連結なグラフ\\
		\item[根] : 木の頂点のうちの1つを定義する\\
		\item[子] : ある頂点について, 隣接する頂点のうち根から遠いもの\\
		\item[親] : ある頂点について, 隣接する頂点のうち根に近いもの\\
		\item[葉] : 木の頂点のうち, 子を持たないもの\\
	\end{description}
\end{block}
}
\onslide*<2>{
	葉ではない各頂点には, 子の値の論理和を格納する.\\
	\begin{itemize}
	\item ある頂点の値が$1$の場合, \\その頂点の下にある葉の\structure{少なくとも1つの頂点は値が$1$}.\\
	\item 根から深さ毎に左から配列に詰めて保持することで\\各頂点に\structure{ランダムアクセスが可能}となる.\\
	\begin{itemize} \item 赤字は格納されている配列の添字 \end{itemize}
	\end{itemize}
}
\onslide*<3>{
	二分木中のある頂点の添字を$i$とすると, \\以下のように他の頂点の添字を求めることができる.\\
	\begin{description}[根 ] 
		\item[根] : 先頭なので$1$\\
		\item[子] : $2\times i + 1$と$2\times i + 2$\\
		\item[親] : $\lfloor \frac{i}{2} \rfloor$ (以後$\frac{r}{m}$が整数でない場合これを切り捨てた値とする)\\
		\item[葉] : 集合$U$の要素$v$に対応する葉の添字は$i + u$\\
	\end{description}
}
\onslide*<2->{
	\subfile{resources/binarytree}
}

\end{frame}
\end{document}
