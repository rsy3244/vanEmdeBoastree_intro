\documentclass[main]{subfiles}

\setcounter{section}{2}
\begin{document}

\begin{frame}\frametitle{操作$\Member(V,x)$}
\begin{itemize}
\item $\Member(V,x)$は, $x$が集合$V$に存在するかを真偽値で返す.\\
\end{itemize}
\begin{columns}[c]
	\begin{column}{0.5\linewidth}
		\tikzset{
			a3/.style={elem, fill=red!20},
		}
		\subfile{resources/intro-set}
	\end{column}
	\begin{column}{0.49\linewidth}
		\begin{itemize}
			\item $\Member(V,5) = \TRUE$\\
			\item $\Member(V,6) = \FALSE$\\
		\end{itemize}
	\end{column}
\end{columns}
\begin{alertblock}{各関数の引数}
関数に渡す引数は要素として取りうる値のみとする
\end{alertblock}
\end{frame}
\end{document}
