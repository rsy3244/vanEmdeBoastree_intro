\documentclass[main]{subfiles}

\setcounter{section}{2}
\begin{document}

\begin{frame}\frametitle{van Emde Boas tree とは\uniqnum{2}}
	van Emde Boas tree (以下vEB木)は\structure{動的集合}を扱うデータ構造
\onslide*<1>{
	\begin{block}{動的集合}
		動的集合とは, 集合に対して後述の$\Member(V,x)$, $\Insert(V,x)$, $\Delete(V,x)$が行えるようなデータ構造
	\end{block}
	\begin{itemize}
		\item 今回扱う要素は\structure{非負整数}とする\\
		\item vEB木が保持しうる要素の集合を\structure{全体集合$U$}とし, \\
			その大きさを\structure{$u$}とする
		\item vEB木が現在保持している集合を\structure{$V$}とし, \\
			その大きさを\structure{$n$}とする
	\end{itemize}
	
}
\onslide*<2>{
	\begin{block}{操作}
		\begin{description}
		\setlength{\labelwidth}{16ex}
		\setlength{\itemindent}{8ex}
		%ここよろし?
		\item[$\Member(V,x)$]		$V$に$x$が存在するかを返す\\
		\item[$\Min(V)$]			$V$の要素の最小値を返す\\
		\item[$\Max(V)$]			$V$の要素の最大値を返す\\
		\item[$\Successor(V,x)$]	$V$の$x$より大きい最小の要素を返す\\
		\item[$\Predecessor(V,x)$]	$V$の$x$より小さい最大の要素を返す\\
		\item[$\Insert(V,x)$]		$V$に$x$を挿入する\\
		\item[$\Delete(V,x)$]		$V$から$x$を削除する\\
		\end{description}
	\end{block}
	\begin{block}{}
		vEB木では,これらの操作が最悪時間計算量\structure{$O(\log\log u)$で実行可能\\}
	\end{block}
}
\end{frame}

\end{document}
