\documentclass[main]{subfiles}

\setcounter{section}{2}
\begin{document}

\begin{frame}\frametitle{操作: $\Successor(V,x), \Predecessor(V,x)$}
\begin{itemize}
	\item $\Successor(V,x)$は, 集合$V$の$x$より大きい最小の要素を返す.\\
	\item $\Predecessor(V,x)$は, 集合$V$の$x$より小さい最大の要素を返す.\\
\end{itemize}
\begin{tikzpicture}[overlay, remember picture]
	\node[fill=red!20,rectangle, minimum size=1em] at ($(pic cs:Sucval) + (.1,.1)$) {};
	\node[fill=green!20,rectangle, minimum size=1em] at ($(pic cs:Preval) + (.1,.1)$) {};
\end{tikzpicture}

\begin{columns}[c]
	\begin{column}{0.5\linewidth}
		\tikzset{
			a2/.style={elem, fill=red!20},
			a3/.style={elem, fill=green!20},
		}
		\subfile{resources/intro-set}
	\end{column}
	\begin{column}{0.49\linewidth}
		\begin{itemize}
			\item $\Successor(V,2) = \tikzmark{Sucval}3$\\
			\item $\Predecessor(V,7) = \tikzmark{Preval}5$\\
		\end{itemize}
	\end{column}
\end{columns}
\end{frame}
\end{document}
