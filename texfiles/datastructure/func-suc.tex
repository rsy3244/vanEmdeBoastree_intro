\documentclass[main]{subfiles}

\setcounter{section}{2}
\begin{document}

\begin{frame}\frametitle{操作$\func{successor}(e)$, $\func{predecessor}(e)$}
\begin{itemize}
	\item $\func{successor}(e)$は, 集合の要素の最小値を返す.\\
	\item $\func{predecessor}(e)$は, 集合の要素の最大値を返す.\\
\end{itemize}

\begin{columns}[c]
	\begin{column}{0.5\linewidth}
		\tikzset{
			a4/.style={circle, fill=red!20},
		}
		\subfile{resources/intro-set}
	\end{column}
	\begin{column}{0.49\linewidth}
		$\func{min}() = 2$\\
		$\func{max}() = 11$\\
	\end{column}
\end{columns}
\end{frame}
\end{document}
