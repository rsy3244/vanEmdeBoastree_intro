\documentclass[main]{subfiles}

\setcounter{section}{2}
\begin{document}

\begin{frame}\frametitle{操作$\func{delete}(e)$}
\begin{itemize}
	\item $\func{}(delete)$は, 集合から$e$を削除する.\\
\end{itemize}

\begin{columns}[c]
	\begin{column}{0.5\linewidth}
		\subfile{resources/intro-set-inserted}
	\end{column}
	\begin{column}{0.49\linewidth}
		\subfile{resources/intro-set-ps}
	\end{column}
\end{columns}
\begin{alertblock}{各関数の引数}
関数に渡す引数は要素として取りうる値のみとする
\end{alertblock}
\begin{tikzpicture}[overlay, remember picture]
\draw[->, very thick] (5,4) to (7,4);
\end{tikzpicture}
\end{frame}
\end{document}
