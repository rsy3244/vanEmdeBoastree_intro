%%%% 和文用 %%%%%
\usepackage{bxdpx-beamer}
\usepackage{pxjahyper}
\usepackage{minijs}%和文用
\usepackage{latexsym}
\usepackage{tikz}
\usetikzlibrary{calc,quotes,positioning}
\renewcommand{\kanjifamilydefault}{\gtdefault}%和文用

%%%% スライドの見た目 %%%%%
\usetheme{Madrid}
\usefonttheme{professionalfonts}
\setbeamertemplate{frametitle}[default][center]
\setbeamertemplate{navigation symbols}{}
\setbeamercovered{transparent=0}%好みに応じてどうぞ)
\setbeamertemplate{footline}[page number]
\setbeamerfont{footline}{size=\normalsize,series=\bfseries}
\setbeamercolor{footline}{fg=black,bg=black}
%%%%

%%%% 定義環境 %%%%%
\usepackage{amsmath,amssymb}
\usepackage{amsthm}
\theoremstyle{definition}
\newtheorem{theorem}{定理}
\newtheorem{definition}{定義}
\newtheorem{proposition}{命題}
\newtheorem{lemma}{補題}
\newtheorem{corollary}{系}
\newtheorem{conjecture}{予想}
\newtheorem*{remark}{Remark}
\renewcommand{\proofname}{}
%%%%%%%%%

%%%%% フォント基本設定 %%%%%
\usepackage[T1]{fontenc}%8bit フォント
\usepackage{textcomp}%欧文フォントの追加
\usepackage[utf8]{inputenc}%文字コードをUTF-8
\usepackage{otf}%otfパッケージ
\usepackage{txfonts}%数式・英文ローマン体を txfont にする
\usepackage{bm}%数式太字
%%%%%%%%%%

\newcommand{\func}[1]{\ensuremath\mathrm{#1}}
 
\title[略タイトル]{van Emde Boas Trees}
\author[Mitsuyoshi]{光吉 健汰}
\institute[IKN]{北海道大学工学部 情報エレクトロニクス学科 情報理工学コース 3年\\
	情報知識ネットワーク研究室}
\date{\today}%日付
